% template.tex, dated April 5 2013
% This is a template file for Annual Reviews 1 column Journals
%
% Compilation using ar-1col.cls' - version 1.0, Aptara Inc.
% (c) 2013 AR
%
% Steps to compile: latex latex latex
%
% For tracking purposes => this is v1.0 - Apr. 2013

\documentclass{style/ar-1col}
\usepackage{url}
\usepackage[numbers]{natbib}

\setcounter{secnumdepth}{4}

% Metadata Information
\jname{Xxxx. Xxx. Xxx. Xxx.}
\jvol{AA}
\jyear{YYYY}
\doi{10.1146/((please add article doi))}


% Document starts
\begin{document}

% Page header
\markboth{Author et al.}{Short title}

% Title
\title{Title: Subtitle}


%Authors, affiliations address.
\author{Author B. Authorone,$^1$ Firstname C. Authortwo,$^2$ and D. Name Authorthree$^3$
\affil{$^1$Department/Institute, University, City, Country, Postal code; email: author@email.edu}
\affil{$^2$Department/Institute, University, City, Country, Postal code}
\affil{$^3$Department/Institute, University, City, Country, Postal code}}

%Abstract
\begin{abstract}
Abstract text, approximately 150 words. 
\end{abstract}

%Keywords, etc.
\begin{keywords}
keywords, separated by comma, no full stop, lowercase
\end{keywords}
\maketitle

%Table of Contents
\tableofcontents


% Heading 1
\section{INTRODUCTION}
Please begin the main text of your article here. 



%Heading 1
\section{FIRST-LEVEL HEADING}
This is dummy text. 
% Heading 2
\subsection{Second-Level Heading}
This is dummy text. This is dummy text. This is dummy text. This is dummy text.

% Heading 3
\subsubsection{Third-Level Heading}
This is dummy text. This is dummy text. This is dummy text. This is dummy text. 

% Heading 4
\paragraph{Fourth-Level Heading} Fourth-level headings are placed as part of the paragraph.

%Example of a Figure
\section{ELEMENTS\ OF\ THE\ MANUSCRIPT} 
\subsection{Figures}Figures should be cited in the main text in chronological order. This is dummy text with a citation to the first figure (\textbf{Figure \ref{fig1}}). Citations to \textbf{Figure \ref{fig1}} (and other figures) will be bold. 

\begin{figure}[h]
\includegraphics[width=3in]{SampleFigure}
\caption{Figure caption with descriptions of parts a and b}
\label{fig1}
\end{figure}

% Example of a Table
\subsection{Tables} Tables should also be cited in the main text in chronological order (\textbf {Table \ref{tab1}}).

\begin{table}[h]
\tabcolsep7.5pt
\caption{Table caption}
\label{tab1}
\begin{center}
\begin{tabular}{@{}l|c|c|c|c@{}}
\hline
Head 1 &&&&Head 5\\
{(}units)$^{\rm a}$ &Head 2 &Head 3 &Head 4 &{(}units)\\
\hline
Column 1 &Column 2 &Column3$^{\rm b}$ &Column4 &Column\\
Column 1 &Column 2 &Column3 &Column4 &Column\\
Column 1 &Column 2 &Column3 &Column4 &Column\\
Column 1 &Column 2 &Column3 &Column4 &Column\\
\hline
\end{tabular}
\end{center}
\begin{tabnote}
$^{\rm a}$Table footnote; $^{\rm b}$second table footnote.
\end{tabnote}
\end{table}

% Example of lists
\subsection{Lists and Extracts} Here is an example of a numbered list:
\begin{enumerate}
\item List entry number 1,
\item List entry number 2,
\item List entry number 3,\item List entry number 4, and
\item List entry number 5.
\end{enumerate}

Here is an example of a extract.
\begin{extract}
This is an example text of quote or extract.
This is an example text of quote or extract.
\end{extract}

\subsection{Sidebars and Margin Notes}
% Margin Note
\begin{marginnote}[]
\entry{Term A}{definition}
\entry{Term B}{definition}
\entry{Term C}{defintion}
\end{marginnote}

\begin{textbox}[h]\section{SIDEBARS}
Sidebar text goes here.
\subsection{Sidebar Second-Level Heading}
More text goes here.\subsubsection{Sidebar third-level heading}
Text goes here.\end{textbox}



\subsection{Equations}
% Example of a single-line equation
\begin{equation}
a = b \ {\rm ((Single\ Equation\ Numbered))}
\end{equation}
%Example of multiple-line equation
Equations can also be multiple lines as shown in Equations 2 and 3.
\begin{eqnarray}
c = 0 \ {\rm ((Multiple\  Lines, \ Numbered))}\\
ac = 0 \ {\rm ((Multiple \ Lines, \ Numbered))}
\end{eqnarray}

% Summary Points
\begin{summary}[SUMMARY POINTS]
\begin{enumerate}
\item Summary point 1. These should be full sentences.
\item Summary point 2. These should be full sentences.
\item Summary point 3. These should be full sentences.
\item Summary point 4. These should be full sentences.
\end{enumerate}
\end{summary}

% Future Issues
\begin{issues}[FUTURE ISSUES]
\begin{enumerate}
\item Future issue 1. These should be full sentences.
\item Future issue 2. These should be full sentences.
\item Future issue 3. These should be full sentences.
\item Future issue 4. These should be full sentences.
\end{enumerate}
\end{issues}

%Disclosure
\section*{DISCLOSURE STATEMENT}
If the authors have noting to disclose, the following statement will be used: The authors are not aware of any affiliations, memberships, funding, or financial holdings that
might be perceived as affecting the objectivity of this review. 

% Acknowledgements
\section*{ACKNOWLEDGMENTS}
Acknowledgements, general annotations, funding.

% References
%
% Margin notes within bibliography
\section*{LITERATURE\ CITED}

To download the appropriate bibliography style file, please see \url{http://www.annualreviews.org/page/authors/author-instructions/preparing/latex}. 

\noindent
Please see the Style Guide document for instructions on preparing your Literature Cited.

The citations should be numbered in alphabetical order, with titles. For example:






\begin{verbatim}
\begin{thebibliography}{00}
\bibitem[Acevedo \& Fitzjarrald(2001)]{Acevedo:01}
Acevedo O, Fitzjarrald D. 2001.
The early evening surface-layer transition: temporal and spatial variability.
\textit{J. Atmos. Sci.} 58:2650--67

\bibitem[Acevedo et~al.(2009)Acevedo, Moraes, Degrazia, Fitzjarrald, Manzi \&
  Campos]{Acevedo:09}
Acevedo O, Moraes O, Degrazia G, Fitzjarrald D, Manzi A, Campos J. 2009.
Is friction velocity the most appropriate scale for correcting nocturnal carbon
  dioxide fluxes?
\textit{Agric. For. Meteorol.} 149:1--10

\bibitem[Baas et~al.(2006)Baas, Steeneveld, {van de Weil} \& Holtslag]{Baas:09}
Baas P, Steeneveld G, {van de Weil} B, Holtslag A. 2006.
Exploring self-correlation in the flux-gradient relationships for stably
  stratified conditions.
\textit{J. Atmos. Sci.} 63:3045--54

\bibitem[Badran, Thiria \& Crepon(1991)]{Badran:91}
Badran F, Thiria S, Crepon M. 1991.
Wind ambiguity removal by the use of neural network techniques.
\textit{J. Geophys. Res.} 96:20,521--29

\bibitem[Bakas \& Ioannou(2007)]{Bakas:07}
Bakas NA, Ioannou PJ. 2007.
Momentum and energy transport by gravity waves in stochastically driven
  stratified flows. {Part I}: radiation of gravity waves from a shear layer.
\textit{J. Atmos. Sci.} 64:1509--29

\bibitem[Calanca, Forrer \& Rotach(1998)]{Calanca:98}
Calanca P, Forrer J, Rotach M. 1998.
Toward an integral formulation of the turbulent transfer in a stably stratified
  boundary layer over an ice sheet.
\textit{Q. J. R. Meteorol. Soc.} 124:1--18

\bibitem[D'Asaro \& Lien(2000)]{DAsaro:00}
D'Asaro EA, Lien RC. 2000.
The wave-turbulence transition for stratified flows.
\textit{J. Phys. Oceanog.} 30:123--45

\bibitem[de~Silva et~al.(1996)de~Silva, Fernando, Eaton \& Hebert]{deSilva:96}
de~Silva I, Fernando H, Eaton F, Hebert D. 1996.
Evolution of kelvin-helmholtz billows in nature and laboratory.
\textit{Earth Planetary Sci. Let.} 143:217--31



\end{thebibliography}
\end{verbatim}




\begin{thebibliography}{00}

\bibitem[Acevedo \& Fitzjarrald(2001)]{Acevedo:01}
Acevedo O, Fitzjarrald D. 2001.
The early evening surface-layer transition: temporal and spatial variability.
\textit{J. Atmos. Sci.} 58:2650--67

\bibitem[Acevedo et~al.(2009)Acevedo, Moraes, Degrazia, Fitzjarrald, Manzi \&
  Campos]{Acevedo:09}
Acevedo O, Moraes O, Degrazia G, Fitzjarrald D, Manzi A, Campos J. 2009.
Is friction velocity the most appropriate scale for correcting nocturnal carbon
  dioxide fluxes?
\textit{Agric. For. Meteorol.} 149:1--10

\bibitem[Baas et~al.(2006)Baas, Steeneveld, {van de Weil} \& Holtslag]{Baas:09}
Baas P, Steeneveld G, {van de Weil} B, Holtslag A. 2006.
Exploring self-correlation in the flux-gradient relationships for stably
  stratified conditions.
\textit{J. Atmos. Sci.} 63:3045--54

\bibitem[Badran, Thiria \& Crepon(1991)]{Badran:91}
Badran F, Thiria S, Crepon M. 1991.
Wind ambiguity removal by the use of neural network techniques.
\textit{J. Geophys. Res.} 96:20,521--29

\bibitem[Bakas \& Ioannou(2007)]{Bakas:07}
Bakas NA, Ioannou PJ. 2007.
Momentum and energy transport by gravity waves in stochastically driven
  stratified flows. {Part I}: radiation of gravity waves from a shear layer.
\textit{J. Atmos. Sci.} 64:1509--29

\bibitem[Calanca, Forrer \& Rotach(1998)]{Calanca:98}
Calanca P, Forrer J, Rotach M. 1998.
Toward an integral formulation of the turbulent transfer in a stably stratified
  boundary layer over an ice sheet.
\textit{Q. J. R. Meteorol. Soc.} 124:1--18

\bibitem[D'Asaro \& Lien(2000)]{DAsaro:00}
D'Asaro EA, Lien RC. 2000.
The wave-turbulence transition for stratified flows.
\textit{J. Phys. Oceanog.} 30:123--45

\bibitem[de~Silva et~al.(1996)de~Silva, Fernando, Eaton \& Hebert]{deSilva:96}
de~Silva I, Fernando H, Eaton F, Hebert D. 1996.
Evolution of kelvin-helmholtz billows in nature and laboratory.
\textit{Earth Planetary Sci. Let.} 143:217--31 

\end{thebibliography}


\end{document}
