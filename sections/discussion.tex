\section{Discussion}

The term \emph{pangenome} has previously implied the study of gene families within a given species or clade.
Technological change, in the form of improved sequencing and assembly algorithms, suggests the possibility of building pangenomes that represent collections of genomes, rendering pangenomics a precise study of the evolutionary relationships between whole genomes.
New techniques are being developed and explored to utilize this powerful prior information about genomic variability in a given species or clade.

Often, these methods rely on graph-based representations of pangenomes which capture both sequence and variation between represented genomes.
These methods typically provide the highest performance and accuracy when working with pangenome models.
The consistency of this trend, and the long history of these structures in bioinformatics, suggests that they are likely to remain important.

However, it is not clear that graphical pangenome models will themselves replace linear reference systems.
None of the methods which we have reviewed makes a strong case that the reference system itself should become a graph.
For instance, only a handful of mapping and variant calling methods (primarily those based on variation graphs) even produce alignments or genotype calls in the context of the graph, with the majority reporting them against a linear reference sequence.
In combining sequences with their alignments, graphical pangenomes confuse the traditional concepts of genome position and annotation which are essential for standard research practice.
To date, there is no widely-accepted mechanism to generalize such concepts to graphs.

We speculate that the status quo of genome positions on linear sequences may continue long into the future, even if graphical pangenome models become essential to many kinds of analysis.
On their own, pangenome graphs do not represent any directly measurable aspect of a biological system, and thus their construction and design is guided more by application than any kind of ground truth.
A particular alignment represented in a pangenome graph is a specific interpretation of the given sequence data.
In this view, pangenome graphs are technical artifacts important for analysis, but may not provide a stable foundation for many ``legacy'' techniques.

Linear sequence models provide straightforward ways of thinking about positions, and are completely compatible with graphical models which embed them, suggesting that we may return to this basis even when working with graphical interpretations of pangenomes.
%They remain stable and immutable regardless of any specific alignment that we use to relate them.
%We may need to be able to relate multiple such interpretations during a particular analysis.
%Mechanisms that allow us to support and relate multiple such interpretations will provide greater flexibility when used to support our inqueries 
%A useful pangenome  simply a collection 
A collection of verifiably contiguous sequences found in the set of individuals under study is sufficient to support any of the pangenomic methods we have covered in this review.
%Although, due to high sequencing and analysis costs, this result has rarely been fully realized, it is the logical aim of pangenome surveys.
The manner in which we combine these sequences into a single compressed object that embeds their likely evolutionary relationship, heterozygosity, and ambiguity is highly dependent on our downstream applications.

Precision pangenomic methods, regardless of the specifics of their representation of the pangenome, aim to provide accurate and unbiased access to this collection of sequences with minimal resource costs.
These methods stand to become essential to genome science as coherent and efficient means to interface with ever larger, and more complete collections of genomes.

%Beyond the scope of current research lie methods that address the pangenome itself, the development of 

%Here, we have provided a review of recent work in this nascent field.


%We have presented recent work on pangenomic methods based on lossless pangenomic models.

%JME: following a recommendation from Jonas, I'm leaving this paragraph in discussion for possible inclusion later

%Among the graph mappers for variation graphs, a theme has emerged in their accuracy. 
%They typically do not improve mapping accuracy over the entire genome compared to state-of-the-art linear mappers. 
%However, they do significantly improve read mapping accuracy \emph{over variants} \cite{Garrison_2019, Rakocevic_2019, Kim_2019}. 
%In this, we see a concrete example of pangenomic methods mitigating reference bias. 
%%The unimproved (and often slightly diminished) performance over the rest of the genome probably indicates a combination of the relative nascency of the graph mapping tools and the burden of more complicated computational heuristics.
%\todo{JAS: Would the above paragraph fit better in the discussion?}

\begin{comment}
  % figure out how to use this
In precision medicine, we seek accurate inference of a given patient genome.
Improving our prior model for what sequences we expect to see will reduce the time and cost required to infer a patient genome.
This does not only help us when we are genotyping known variants.
If most of the sequence and variation in any given individual is also found in the reference pangenome which we use, then we are left with a smaller set of sequences that do not map to the reference to consider when we attempt to infer novel variation.
\end{comment}


\begin{comment}

% Summary Points
\begin{summary}[SUMMARY POINTS]
\begin{enumerate}
\item Summary point 1. These should be full sentences.
\item Summary point 2. These should be full sentences.
\item Summary point 3. These should be full sentences.
\item Viz Discussion: SH: I think it is important to say, that there are such a low number of tools out there specialized in visualizing genome graphs, they are not even classified in a recent genomic data visualization review by Nusrat et al. \cite{Nusrat_2019}
\end{enumerate}
\end{summary}

% Future Issues
\begin{issues}[FUTURE ISSUES]
\begin{enumerate}
\item Future issue 1. These should be full sentences.
\item Future issue 2. These should be full sentences.
\item Future issue 3. These should be full sentences.
\item Future issue 4. These should be full sentences.
\end{enumerate}
\end{issues}

\end{comment}
