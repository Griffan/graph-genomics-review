\section{Building a pangenome}

A collection of genomic sequences, either raw reads or assembled genomes, can itself provide a pangenomic model.
But this model lacks a mechanism to define the relationships between the sequences.
It cannot assert anything more than the presence of a given sequence in the pangenome.
In practice, we need to know when a particular sequence is identical to another by descent from a common ancestor.
This information suggests phylogenetic structures that result from evolutionary processes, which can in turn help support our understanding of 

Furthermore, the identity of sequences by descent provides a basis for the compact representation of pangenomes.
Our model need not grow linearly with the number of genomes that it includes.
Rather, we should obtain improved compression of individual genomes in the context of a pangenome.
Shared segments of DNA only need to be represented once.
If most DNA is shared, then this makes the unique sequences of the pangenome much smaller than the total sequences in the genomes that we use to build it, providing compression that is both practical and helpful for analyses.

Methods to construct pangenomic data structures mirror the classes of pangenomic models.
A pangenome may simply be a collection of sequences, in which case construction is similar to the genome or metagenome assembly problem.
Or, it may include information about the alignment of sequences.
This alignment could be compressed into a set of variants found against a set of reference sequences.
If this alignment is based on $k$-mers, then it implies a de Bruijn graph, and various methods exploit this to implement efficient or scalable pangenome model construction.
If the alignments is a complete, gapped alignment, covering small and large variation, then the pangenome model can be thought of as a whole genome alignment.
Methods for pangenome construction often use a mixture of approaches, but in most cases their aim is to emulate the full all-to-all alignment of the base sequences.

%The great majority of pangenome construction methods generate or use graphical models
% new organization
% 1) bags of sequences -- pantools comes close (collection of genes), HUPAN is similar with its core/accessory breakdown
% 2) DBG based models (TwoPaCo, LOGAN, McCortex, Bifrost, PRG, Novograph?)
% 3) alignment and match based systems (cactus, REVEAL, vg msga, seqwish)

\subsection{Collecting sequences}

% oldschool pan-genomics
A pangenome can be represented as a collection of sequences.
Determining which sequences provide a compact represenation of a pangenome is non-trivial, and a number of software systems have been developed to support this process.

Panseq \cite{Laing_2010} finds novel regions, determines the core and accessory genome, finds SNPs within the core pangenome, and then determines a subset of loci useful for molecular fingerprinting.
PGAP \cite{Zhao_2011} extends Panseq's approach with modules for evolutionary and functional analysis, and is implemented as a single standalone executable.
% pangenome using a collection of sequences model
Recent work has focused on scaling these techniques to ever larger genomes.
PanTools \cite{Sheikhizadeh_Anari_2018} detects and annotates homology groups in large collections of large genomes using a $k$-mer based approach.
Its detailed graph database model connects the pan-proteome defined by homology groups to genomic annotations and sequences.
HUPAN \citep{Duan_2019} extends the sequence collection model to human and large eukaryotic genomes, taking assembled genomes as input and finding non-reference sequences within them by comparison to a reference genome.

Many published pangenomes are established with ad-hoc methods to determine novel sequences.
These approaches typically apply assembly methods to determine contigs representative of the pangenome, and then align data from individual biosamples back to them to determine their significicance.
Standard alignment and variant calling methods typically work without issue on a pangenome stored as a collection of sequences, and can be applied in custom pipelines tailored to particular species contexts.

\subsection{Adding variation}

Rather than collecting unique sequences that represent a collection of genomes, we can consider small variants between the collection and a reference genome.
Such a model directly implies a directed acyclic graph, ordered along the reference genome, with bubbles at the sites of variation.
It has been key to the first wave of precision pangenomic methods.
%These ``reference and variant'' models cannot compactly represent complex structural variation (such as inversions and copy number amplifications).
%But this has been largely irrelevant to the field.
%The low-cost short read sequencing that is used for pangenomic surveys does not easily support their detection, and so much of the population scale pangenomic information made available in the last decade has been for small variants that comfortably fit into this model.

%A number of methods have been developed around this pangenomic model, but typically they use a specialized or limited version of the model as in internal data structure to drive improvements in read alignment and genome inference.
These methods typically use a specialized or restricted version of the ``reference plus variation'' pangenome graph.
GenomeMapper \cite{Schneeberger_2009} first introduced the linear \emph{genome graph} a reference system for resequencing.
To facilitate its alignment model (which has exponential costs in the number of traversed bubbles), it breaks the graph into blocks of a fixed reference length (typically 256bp).
The Population Reference Graph (PRG) model \cite{dilthey2015improved}, which was validated primarily in the human MHC region, takes a more curated approach to graph construction.
Over multiple phases it builds an acyclic genome graph from $k$-mer based alignment of long haplotypes, additional gene alleles, and finally small variants.
To support its alignment (which has exponential costs in the number of bubbles) and indexing systems, gramtools \cite{Maciuca_2016} combines variants found within a given window length (~30bp) into a single locus.
PanVC \cite{Valenzuela_2018} employs a gapped MSA as its pangenomic model, inferring underlying recombinant haplotype that may then be used as a reference for alignment and variant calling with standard tools.
Seven Bridges' Graph Genome Pipeline \cite{Rakocevic_2019} was vetted on a graph with common variation and indels, which is used primarily to improve read alignment, with alignments reported against the linear reference.
The variation graph toolkit, specifically \textit{vg construct} \cite{Garrison_2018}, can be applied to transform VCF files and reference sequences into genome graphs, which can be rendered into GFA format and are subsequently used by other tools for alignment and genome annotation.

We observe out that most of these methods (excluding GenomeMapper and vg) produce reference-relative alignments or recombinant haplotypes inferred by matching a read set against their pangenomic model.
Thus, their models are not pangenome graph references in the same sense as linear genome references.

\subsection{Colored and linked de Bruijn graphs}

De Bruijn graph based assemblers can be given a pangenomic quality through the addition of ``colors'' to their nodes ($k$-mers) or unitigs (unbranching components in the graph).
Each color provides a mapping between a specific biosample and a subset of the graph.
Efficiently storing such colors allows for population scale variant calling within the context of the graph, as was first demonstrated with Cortex assembler \cite{Iqbal_2012}.
Recent improvements to colored DBG construction, such as Bifrost \cite{holley2019bifrost}, allow the construction of cDBGs from very large sequence sets (the authors build a pangenome of 118,000 \emph{Salmonella} genomes), and further support efficient update of these pangenomic models.
The feature which makes these methods efficient, the fixed $k$ on which they are based, also limits their resolution of repetitive genomic features.
They also struggle handle long noisy reads produce by third generation sequencing, and are lossy with respect to their input sequences.
In response, several methods embed linking information within the DBG that can be used to reconstruct embedded haplotypes or reads \cite{Bolger_2017,Turner_2018}.

\subsection{Whole genome alignments}

% whole genome alignments
Multiple sequence alignments (predecessor) POA and PO-POA \cite{Lee_2002,Grasso_2004} % only small graphs

Alignment based pangenome models.

% hein1989new, ?
Cactus \cite{Paten_2011} as demonstrated in alignathon \cite{earl2014alignathon}
jst \cite{Rahn_2014}
SplitMEM \cite{Marcus_2014}
cdbg \cite{Baier_2015}
REVEAL \cite{linthorst2015scalable}
TwoPaCo \cite{Minkin_2016}
vg msga \cite{Novak_2017a,Garrison_2018,Garrison_2019}
seq-seq-pan \cite{Jandrasits_2018}
Novograph \cite{Biederstedt2018}
Seqwish\footnote{\url{https://github.com/ekg/seqwish}} \cite{Garrison_2019}


\subsection{Positional systems in pangenomes}

One of the most important uses of conventional reference genomes is to provide a positional system for genomic analyses.
Researchers report many genomic features and annotations relative to positions on a reference genomes: variants, genes, binding sites, epigenetic profiles, homologies, and more.
Despite the increasing prominence of pangenomic methods, there have been relatively few inroads into this particular domain, even though the need for such research has been long recognized \cite{computational2016computational}. 

Part of the reason is certainly that pangenomic coordinate systems cannot provide all of the customary advantages that linear coordinate systems currently do.
In linear references, genomic coordinates are easily interpretable, and they unambiguously indicate both the layout of the sequence and the distance between bases.
Similar schemes cannot do the same on genome graphs \cite{Rand_2017} or in collections of sequences.
We note, however, that positional systems in linear references have ambiguities of their own.
In particular, any sequence missing from the reference has no positional system at all, except by ad hoc use of the reference's system.
Yet, for the time being, this particular ambiguity seems to be counterbalanced by the relative informatic ease of comparing different data.

Despite these challenges, some positional systems have been proposed.
\citeauthor{Rand_2017} \citep{Rand_2017} propose some coordinate systems tailored for the GRCh38 alternative contigs that provide some (but not all) of the positive properties associated with linear coordinates.
They also discuss some options for expressing genomic intervals.
Unfortunately, these options all rely either on significantly more verbose representations or on heuristics to reduce the representation.

Another proposal focuses only on genomic variation \cite{paten2018superbubbles}.
This type of annotation is unique in the context of graphical pangenomics because, unlike other annotations, variation is expressed in the topology of the graph.
Accordingly, these authors use the graph topology as a means to describe variation in a manner that is very general and free of reference bias.
Essentially, the proposal is to use bubble-like structures in the graph, which had been recognized as indicative of variation for some time in genome assembly literature \cite{Iqbal_2012, Onodera_2013}.
This solution is rigorous but also quite mathematically complex.

\subsection{Measuring model utility}

Briefly discuss the ``Genome graphs'' paper, \cite{Novak_2017a}, findings in the vg paper \cite{Garrison_2018}, and FORGe \cite{Pritt_2018}.


\begin{comment}
\subsection{Measuring and improving graph utility}

Because \texttt{vg} provides an integrated toolkit for constructing, manipulating, and using genome graphs, it's a tool well-suited for comparing the efficacy of different genome graph construction techniques \cite{Novak_2017, Garrison_2018}.
In the paper ``Genome Graphs'', \citeauthor{Novak_2017a}\ compared different graph constructions based on their utility in read mapping, variant calling, and general graph character \citep{Novak_2017a} \todo{Does it really make sense to lean so heavily on our own unreviewed, unpublished manuscript in a review?}.
They found that genome graphs in general provide superior read mapping and variant calling to their linear counterparts, and that different construction approaches varied greatly even within each region of the genome.
To quantify this variability, they proposed three normalized graph metrics.
 \emph{Graph compression} is the length of primary reference sequence divided by the sum of the length of the nodes in the graph.
It essentially measures the quantity of alt path sequence represented in the graph with respect to the size of the reference.
The \emph{(base) degree} of the graph measures how much branching occurs in the graph, and the \emph{cut width} measures apparent sequence rearrangement \todo{We don't have any evidence to support these being useful or accepted metrics, though.}.

In the same study, \citeauthor{Novak_2017a}\ demonstrated that their graphs improved mapping precision when more polymorphisms were added to the graph.
As they pointed out, this is a potentially surprising result: they originally thought that adding more sequences similar to the read's target would reduce the likelihood of the read mapping to the correct region.
Instead, adding known variants helped the mapping algorithm distinguish their true mapping location from other, paralogous sequences \citep{Novak_2017a}.

Adding more variants to the graph doesn't always improve mapping precision, however.
FORGe is a tool for modeling the effects of adding a variant to the graph \cite{Pritt_2018}. 
The primary benefit of adding a variant to the graph is that it increases the graph's representation of sequence diversity.
Without variance in the graph, reads with significant sequence dissimilarity from the linear genome won't be mapped to the correct region, or won't be mapped at all.
Sometimes, however, adding a variant can reduce alignment accuracy by increasing the ambiguity of similar sequences already in the graph.
In addition, adding variants to the graph can increase the cost of storing and querying the genome index.
To predict the effects of adding a variant, FORGe scores variants based on their frequency in a population, its proximity to other variants, and how it increases/decreases repetitive sequence in the genome.
It can then output a list of top-scoring variants for use in a graph aligner.
\todo{How do we square the results showing FORGe is needed with the results from our unpublished Graph Genomes manuscript? We don't do that work here.}
\end{comment}


\subsection{Pangenomic data formats}

Pangenomes can be stored as collections of sequences in FASTA format.
Variant calls in VCF format may be added to such a collection to describe small or structural variants found in the pangenome.
Collections of FASTA sequences and alignments between them also imply a pangenomic data structure, which can be viewed in the form of a ``dot plot'' in two dimensions.
This approach breaks down when considering large numbers of related sequences, and is not commonly used as a way to convey a pangenome.

To encode graphical pangenomes, the community frequently employs the GFA\footnote{\url{http://gfa-spec.github.io/GFA-spec/}} format for sequence graph exchange.
GFA was originally designed as an interchange format for assembly methods, but the equivalence between graphical pangenomes and the data structures used by assembly methods allowed it to be repurposed as a pangenomic interchange model.

GFA provides no mechanism to express read alignments.
The vg toolkit has developed the GAM format, which generalizes the data model of the SAM/BAM format to work on graphical reference systems, and this has been taken up by a number of other alignment tools.
However, it lacks a simple text-based equivalent.
The recently-proposed GAF\footnote{\url{https://github.com/lh3/gfatools/blob/master/doc/rGFA.md#the-graph-alignment-format-gaf}} format (which generalizes PAF\footnote{\url{https://github.com/lh3/miniasm/blob/master/PAF.md}} to work on graphs) could serve this purpose.
GAF is related to rGFA\footnote{\url{https://github.com/lh3/gfatools/blob/master/doc/rGFA.md}}, which is a proposed restricted specialization of GFA for reference graphs.


\section{Indexing pangenomes} % was indexing genome graphs

A \emph{text index} maps query strings to their occurrences in the indexed text.
The occurrences are typically reported as a list of starting positions, from which one can easily determine the substrings matching the query.

Indexing the sequences encoded in a graph is more involved.
The number of $k$~bp paths often grows exponentially in $k$, and the number of distinct $k$-mers encoded as path labels may also grow exponentially.
Indexing or even enumerating all $k$-mers in the graph may be infeasible for reasonable values of $k$, and if the starting position of the occurrence reported by the index is in a complex region of the graph, there may be an exponential number of $k$~bp paths to investigate.

In practice, indexes for genome graphs must make trade-offs not encountered in text indexes.
In order to limit the exponential growth, the index may only support relatively short query strings.
Some indexes (e.g.\ \cite{Siren_2014}) support longer queries by doing extensive preprocessing.
In other indexes (e.g.\ \cite{Thachuk_2013,Huang_2013,Maciuca_2016}), queries mapping to complex graph regions can be slow.
Instead of indexing the entire graph, the index may only contain $k$-mers from a simplified graph, or from specific paths of the graph.
While finding the path matching the query may be expensive in some cases, indexes typically save space by only reporting the starting position of the match.

\subsection{Indexing sequences using a graph}

The FM-index \cite{Ferragina_2005} is a text index, based on the Burrows--Wheeler transform (BWT) \cite{Burrows_1994}, that is frequently used with DNA sequences.
One variant of the FM-index, the RLCSA \cite{Maekinen_2010}, run-length encodes the BWT, allowing it to store and index a collection of similar sequences space-efficiently.
\citeauthor{Huang_2010}\ \cite{Huang_2010} observed that if we know a good global alignment of the sequences, we can use that information to make the index both smaller and faster.
\citeauthor{Na_2016}\ \cite{Na_2016,Na_2018} developed this approach further in their FM-index of alignment.
While the articles do not mention it, both \citeauthor{Huang_2010}\ and \citeauthor{Na_2016}\ use the graph induced by the alignment as a space-efficient representation of the sequences.

\subsection{Indexing acyclic graphs}

One class of graph indexing methods supports only acyclic graphs, often represented as directed acyclic graphs (DAGs).
This constraint can exist either because the acyclicity of the graph provides guarantees that simplify the problem, or because incedental features of the method's software implementation preclude use on cyclic graphs.
One common class of DAGs, here called \emph{VCF graphs}, represents each contig as a mostly linear run of graph nodes, branching to allow only for substitutions and possibly insertions and deletions.
\todo{``VCF graph'' is a placeholder; is there a better term for this shape?}

GenomeMapper \cite{Schneeberger_2009} was the first graph-based read aligner.
It builds a VCF graph from a reference sequence and a set of variants.
To index the graph, GenomeMapper uses a simple hash-based $k$-mer index, with $k \le 13$ to limit memory usage.

GCSA \cite{Siren_2014} was the first attempt to use the BWT with graphs.
It either takes a prebuilt DAG or generates one from a multiple alignment of sequences.
GCSA then applies a number of graph transformations that preserve the language, or set of end-to-end sequences, until the nodes can be unambiguously sorted by the labels of the paths starting from the node.
When the complexity of the graph is sufficiently low, these transformations are reasonably fast and do not increase the size of the graph significantly.
However, a phase transition happens at a certain threshold, and the transformed graph quickly becomes too large to handle above the threshold.
\todo{JME: threshold of what? variant density?}

BWBBLE \cite{Huang_2013} is a BWT-based representation for VCF graphs.
Simple substitutions are encoded in the sequence using IUPAC codes, and the sequence is indexed using a normal FM-index.
Because each base can be encoded using 8 different characters, the search branches at every base to cover all possible characters which admit the base searched.
In practice, most branches quickly run out of results and can be pruned from the search.
Insertions and deletions produce separate sequences, with selected amount of context around the variant.
The length of this context is an effective upper bound for query length.

The vBWT \cite{Maciuca_2016} took another approach to using the BWT for indexing VCF graphs.
It encodes variants as \texttt{(ref|alt1|alt2|\dots)} in the sequence.
When the search encounters a variant, it must branch to handle each allele separately.
Both BWBBLE and vBWT trade faster index construction for slower queries.
However, a combination of IUPAC codes for substitutions, the vBWT approach for other variants, and a $k$-mer index for matching the first 5--10 bases, is faster than either of the originals \cite{Buechler_2019}.

\subsection{General graphs}

Some text indexes are based on Lempel--Ziv parsing or context-free grammars.
These indexes first find partial matches between the query string and the indexed phrases and then combine the partial matches into full matches using two-dimensional range queries.
In the hypertext index \cite{Thachuk_2013}, each node is a separate phrase.
Queries mapping to a single node or crossing a single edge can be matched efficiently, while finding mappings to complex graph regions can be slow.

\citeauthor{Bowe_2012}\ \cite{Bowe_2012} used techniques similar to GCSA for representing de~Bruijn graphs.
If the graph transformations used in GCSA construction are stopped after $i$ steps, the resulting graph is equivalent to an order-$2^{i}$ de~Bruijn graph.
This de~Bruijn graph can be used to approximate the original graph.
There are no false negatives, but matches longer than $2^{i}$ may be false positives.
By using this approach, GCSA2 \cite{Siren_2017} attempts to support fast queries in arbitrary graphs.

GCSA2 faces the same issues with complex graphs as GCSA.
In practice, most graphs must be simplified before they can be indexed.
Typical simplifications include removing high-degree nodes and complex regions from the graph and replacing them with the reference sequence.
If a collection of haplotypes is available, the removed regions can be replaced with new subgraphs that contain separate paths for each distinct local haplotype \cite{Siren_2019}.
This way, the index contains all $k$-mers from the haplotypes, while usually missing $k$-mers from their recombinations.

\subsection{Indexing graphs using sequences}

Instead of attempting to index the entire graph, it is often sufficient to index only selected paths in it.
CHOP \cite{Mokveld_2018} takes the paths corresponding to haplotypes and breaks them into smaller pieces.
The distinct pieces form an artificial linear reference, which can be used with any read aligner.
The process guarantees that any substring of the haplotypes of length $k$ is also a substring of one of the pieces.
As with BWBBLE, $k$ represents an effective upper bound for query length.

The ``Pan-genome [sic] Seed Index'' (PSI) \cite{Ghaffaari_2019} follows a similar approach with artificial paths.
Instead of using haplotypes, PSI uses a greedy algorithm to find a set of paths that covers as many $k$~bp windows in the graph as possible.
An index using these paths alone already works well in practice.

When a fully sensitive index is needed, PSI can reverse the role of the query strings and the graph.
While complex graph regions may contain an excessive number of $k$-mers, the reads mapping to them only contain a limited number of $k$-mers.
By indexing a batch of reads and searching for the complex regions in that index, all mappings of the query strings to the graph can be found with reasonable resources.


\subsection{Indexing haplotypes and genomes in variation graphs}
% Erik

Outline progression from PBWT, gpBWT, to GBWT.

%\subsubsection{de Bruijn}

% BOSS: \cite{Bowe_2012} (\cite{Roedland_2013} is similar)

%\subsubsection{VCFs / genotype calls / haplotypes / binary matrices}

%\subsubsection{gPBWT, GBWT}

% Jouni: We should at least include this


%\subsubsection{Alignments / collections of strings}

% Jouni: We probably don't have space for this.



