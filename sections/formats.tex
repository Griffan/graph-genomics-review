\section{Pangenomic data formats}
\label{sec:formats}

A number of common data formats are used to exchange pangenomic models.
Pangenomes can be stored as collections of sequences in FASTA format.
Variant calls in VCF format \cite{danecek2011variant} may be added to such a collection to describe small or structural variants found in the pangenome.
However, to exchange graphical pangenomes, the community frequently uses a subset of version 1 of the Graphical Fragment Assembly format (GFAv1)\footnote{\url{https://github.com/GFA-spec/GFA-spec/blob/master/GFA1.md}}.
Only a small subset of GFA is required to represent pangenome graphs, but using this format allows pangenomic analyses to use many genome assembly tools.

To represent read alignments to pangenome graphs, the \textsc{VG} toolkit has developed the GAM format \cite{Garrison_2018}, which generalizes the SAM/BAM \cite{Li_2009} data model to pangenome graphs.
GAM is produced by several other alignment tools \cite{Rautiainen_2019b,Jain_2019a}, and consumed by numerous downstream applications.
The Graph Alignment Format (GAF)\footnote{\url{https://github.com/lh3/gfatools/blob/master/doc/rGFA.md\#the-graph-alignment-format-gaf}} generalizes the text-based Pairwise Alignment Format (PAF)\footnote{\url{https://github.com/lh3/miniasm/blob/master/PAF.md}} to work on graphs encoded in GFA.
GAF can describe mappings to graphs encoded in rGFA\footnote{\url{https://github.com/lh3/gfatools/blob/master/doc/rGFA.md}}, which is a specialization of GFA for reference pangenome graphs.
